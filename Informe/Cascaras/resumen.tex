%---------------------------------------------------------------------
%
%                      resumen.tex
%
%---------------------------------------------------------------------
%
% Contiene el cap�tulo del resumen.
%
% Se crea como un cap�tulo sin numeraci�n.
%
%---------------------------------------------------------------------

\chapter{Resumen}
\cabeceraEspecial{Resumen}

\begin{FraseCelebre}
\begin{Frase}
No basta tener un buen ingenio, lo principal es aplicarlo bien.
\end{Frase}
\begin{Fuente}
Ren� Descartes
\end{Fuente}
\end{FraseCelebre}

El aprendizaje profundo (conocido en ingl�s como \textit{deep learning}) es una rama de la inteligencia artificial que, debido al �xito de su utilizaci�n en problemas de gran complejidad, se ha vuelto popular y ampliamente desarrollado tanto a nivel empresarial como en el campo de la investigaci�n. Se basa en el dise�o de redes neuronales capaces de aprender a extraer caracter�sticas sobre los datos presentados, para as� lograr con mayor precisi�n tareas de clasificaci�n o regresi�n. La profundidad mencionada en su denominaci�n se debe a que las arquitecturas implementadas suelen componerse de numerosos niveles, con el fin de obtener representaciones de los datos que sean adecuadas para el objetivo planteado sobre la red.

El problema que presenta implementar esta t�cnica es el elevado costo
computacional comprendido en la construcci�n de la arquitectura profunda, que ocurre especialmente cuando se trata con cantidades masivas de datos. Adem�s, el hecho de manejar una gran cantidad de hiper-par�metros y sus posibles combinaciones incrementa la dificultad de encontrar de forma r�pida una red neuronal adecuada para el problema tratado.

En este proyecto se plantea como objetivo general el desarrollo de un framework que ofrezca la posibilidad de entrenar redes neuronales mediante los algoritmos y funcionalidades m�s populares del aprendizaje profundo, y reducir el esfuerzo de dise�o y modelado utilizando herramientas de c�mputo en paralelo. Esto �ltimo implica poder distribuir el trabajo computacional tanto a nivel local sobre los n�cleos de un procesador, como tambi�n sobre los nodos que componen un cl�ster. Adem�s se define como un objetivo espec�fico evaluar la potencialidad del software desarrollado mediante su aplicaci�n en tareas de alta complejidad, particularmente en problemas de clasificaci�n sobre datos de electroencefalograf�a.

\smallskip
\noindent \textbf{Palabras claves:} red neuronal, aprendizaje profundo, autocodificador, c�mputo distribuido, electroencefalograf�a.

\endinput
% Variable local para emacs, para  que encuentre el fichero maestro de
% compilaci�n y funcionen mejor algunas teclas r�pidas de AucTeX
%%%
%%% Local Variables:
%%% mode: latex
%%% TeX-master: "../Tesis.tex"
%%% End:
